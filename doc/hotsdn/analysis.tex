\section{Motivating Examples}

In this section, we charaterize the workloads of two widely deployed service in today's datacenter, highlighting the features influencing the design decisions of $A^2Net$.

\subsection{Shuffle}

MapReduce is a widely deployed service handling many throughput preferred tasks in today's datacenter. 


\subsection{Partition-Aggregator}

Partition-Aggregator \cite{DCTCP} workloads are usually generated by the data center applications handling user requests, e.g. emails, search engine and social networks content composition, which are usually built with soft real-time style for interactive. As shown in Figure \ref{fig:par-agg}, requests from users are split into multiple \emph{sub-requests} by an aggregator in the root layer and delivered to the distributed worker processes in the system via aggregators on different layers. 

To meet the all-up SLA \cite{DCTCP}, aggregators usually assign deadlines to the worker nodes in lower layers. When deadline arrives, the aggregator will directly return the collected results back to the upper layer instead of waiting for all nodes to complete. Those discarded results in lagged behind nodes can ultimately lower the quality of the responses. As we explained in Introduction section, network transfer occupies a considerable part of total workloads of data center applications. The worker deadlines mean that network flows carring requests and responses have deadlines. Only the network flows finished within deadlines can contribute to the throughput of application, on another side, the slow and ultimately discarded flows can waste network bandwidth.

However, the commonly used TCP protocol strives for maximizing network throughput while keeping fairness, being unaware of the deadlines of flows. The lack of prioritization among flows causes that \emph{urgent} flows wait behind latency-insensitive flows just for fairness, making them miss the deadlines. To examine this issue, we reproduce the workload consisting of both throughput-preferred but latency-insensitive flows and latency flows....


\begin{figure}
  \centering
  \includegraphics[width=0.4\textwidth]{pic/placeholder}\\
  \caption{}
  \label{fig:par-agg}
\end{figure}

